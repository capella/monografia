%% ------------------------------------------------------------------------- %%
\chapter{Introduction}
\label{cap:introducao}

\section{Motivation}

Nowadays the term IoT (Internet of Things) is turning a buzzword.  People, government, and industry have learned that connect simple devices to the internet, can produce a lot of useful data. This data can be used to improve production, make houses more smart e save money and time. Some studies show that by 2020 will be more than 20 billion IoT devices connected to the internet \cite{gartner}.

This massive number of devices create new safety and security challenges. However, the vast number of known attacks in the lasts years show the responsible for this devices is not taking this challenge seriously. Some of this attacks can be in seen in the article \cite{attacks}. This kind of attacks can occur inside critical infrastructures, like nuclear plants, health, and transportation system.

Because these devices are connected on the internet, its software, network, and hardware layers can be attack and damage.  Study technics to verify if one device was attacked and test these technics in real hardware are the principal object of this thesis. Attacks will ever occur, but if it occurs, they need to be quickly identified and solved.

The process of remote verify these devices receives the name of Remote Attestation. How to incorporate this feature to an embedded device without increasing its cost is another motivation of these work.

\section{Objective}

The objective of this work is to study and implement from scratch the simple and best remote attestation technic for embedded devices. The technique chosen was SMART \cite{smart}. 

Because the SMART implementation involves many parts, a secondary objective of this work is to explain in detail all the steps need for building a microcontroller with SMART in an FPGA. Also, how to program it.

\section{Structure}

In the first chapter key concepts are introduced. It explains all the topics necessary to understood this work. It also presents the principal articles related to remote attestation. After this chapter, you will be apple the understood all the processes necessary to implement and test a device designed by hardware description language (HDL).

The second chapter will explain how we implement the device running SMART. This chapter will not show how the device was programmed. Instead, it will explain why all project decisions are made.

%falar que atuamente varios sao vulner�veis

%pegamos uma solucao, melhoramos, testamos e vimos que eh possivel

%- o porque o smart

%- parte de hardware e codigo

%- colocamos em um fpga e testamos

%um pouco de cada capitulo

